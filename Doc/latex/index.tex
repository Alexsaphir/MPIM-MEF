\hypertarget{index_Présentation}{}\section{Présentation}\label{index_Présentation}
On utilise des matrices S\+K\+Y\+L\+I\+NE pour représenter les matrices en mémoires. On peut alors en calculer une décomposition de Cholesky. Enfin, on peut résoudre le problème d\textquotesingle{}élément finis. \hypertarget{index_Cholesky}{}\section{Cholesky}\label{index_Cholesky}
Ici on ne stocke pas la matrice L calculé, mais sa transposée. Cela permet de n\textquotesingle{}avoir a gérer que des acces en ligne durant les phases d\textquotesingle{}écriture. \hypertarget{index_SOLVER}{}\section{S\+O\+L\+V\+ER}\label{index_SOLVER}
Il faut modifier A(x) et F(x) pour résoudre le problème voulu. \hypertarget{index_Compilation}{}\section{Compilation}\label{index_Compilation}
Pour compiler le code executer le script {\itshape compile.\+sh}. 